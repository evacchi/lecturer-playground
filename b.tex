% This is a demonstration file distributed with the
% Lecturer package (see lecturer-doc.pdf).
%
% You can recompile the file with a basic TeX implementation,
% using pdfTeX or LuaTeX with the plain format.
% 
% The reusable part ends somewhere around line 300.
%
% Author: Paul Isambert.
% Date: July 2010. 


\input{luatex85.sty}
\input{luaotfload.sty}
\input lecturer

\font\maintitlefont    = {name: League Gothic} at 70pt
\font\slidetitlefont   = {name: Reklame Script DEMO} at 30pt
\font\slidetitlefont   = {name: Fira Sans} at 20pt
\font\slidetitlefont   = {name: League Gothic} at 30pt
\font\subtitlefont     = {name: Reklame Script DEMO}
\font\mainsubtitlefont = {name: Reklame Script DEMO} at 30pt
\font\titlefont        = {name: Fira Sans}
\font\mainfont         = {name: Fira Sans}

\newcolor{darkred}{rgb}{.6 0 0}
\newcolor{lightgrey}{grey}{.85}
\newcolor{darkgrey}{grey}{.7}



\setparameter fresh-cut-day:
  tropiteal     = "0.00  0.66  0.78"
  teal-drop     = "0.25  0.75  0.80"
  whitetrash    = "0.98  0.95  0.91"
  atomic-bikini = "0.68  0.89  0.22"
  feeble-week   = "0.56  0.75  0.00"
  black         = " 0 0 0 "

\setparameter palette:
  first-slide-background = tropiteal
  first-slide-title      = whitetrash
  first-slide-subtitle   = whitetrash
  slide-title-foreground = feeble-week
  slide-foreground       = black
  slide-background       = whitetrash
  item-color             = tropiteal

%\updatepalette[fresh-cut-day]
%tropiteal
\newcolor{first-slide-background}{rgb}{0.00  0.66  0.78}
\newcolor{first-slide-title}{rgb}{0.98  0.95  0.91}
\def\setpalettekey#1#2#3{%
  \newcolor{#1}{rgb}{
    \usevalue #2 : #3
  }
}

\setpalettekey{first-slide-background}{fresh-cut-day}{tropiteal}
\setpalettekey{first-slide-title}{fresh-cut-day}{whitetrash}
\setpalettekey{first-slide-subtitle}{fresh-cut-day}{whitetrash}
\setpalettekey{slide-title-foreground}{fresh-cut-day}{feeble-week}
\setpalettekey{slide-foreground}{fresh-cut-day}{black}
\setpalettekey{slide-background}{fresh-cut-day}{whitetrash}
\setpalettekey{item-color}{fresh-cut-day}{tropiteal}



\setparameter job:
  mode           = presentation % The handout is ready, just turn it on.
  author         = "Edoardo Vacchi"
  title          = "History of a Value"
  date           = "May, 4 2018"
  fullscreen     = true
  autofullscreen = false
  font           = \mainfont % Plain TeX's default, but must be specified
                          % to set it back after a \step with another font.
  
\setparameter slide:
  everyslide    = \everyslide
  hsize         = ".6\pdfpagewidth"
  top           = 76pt
  bottom        = 42pt
  width         = 16cm
  height        = 9cm
  areas*        = "title author date"
  bookmarkstyle = italic
  baselineskip  = 12pt
  topskip       = \baselineskip

\def\everyslide{%
  \position{titlebar}\slidetitle
}

%
% Top and bottom bars.
% 

\setarea{topleft topright bottomleft bottommiddle bottomright}
  height = 18pt

\setarea{titlebar}
  vshift     = 18pt
  left       = 1em
  height     = 40pt
  foreground = slide-title-foreground
  topskip    = 30pt
  hpos       = fr
  font       = \slidetitlefont

% bottomleft is already fully specified.
  

%
% The maths on the right.
%
\setarea{math}
  width      = ".2\pdfpagewidth"
  hshift*    = 0pt
  vshift     = 58pt
  vshift*    = 18pt
  foreground = "grey .75"
  hpos       = fr
  vpos       = center
  baselineskip = 18pt


%
% The title slide.
%
\setslide{titleslide}
  areas      = "title topleft topright"
  bookmark   = false
  foreground = first-slide-title
  background = first-slide-background
  hpos       = rr
  everyslide = {}

\setarea{title}
  hshift*    = 1cm
  vshift     = 2.5cm
  height     = 45pt
  topskip    = 25pt
  top        = 1em
  %frame      = "width=.2em, corner=round, color=darkgrey"
  hpos       = rr
  font       = \maintitlefont
  foreground = first-slide-title

\setarea{author}
  hshift*    = 5cm
  vshift     = 6cm
  height     = 28pt
  topskip    = 20pt
  foreground = white
  frame      = "width=.5em, corner=round"
  hpos       = rr
  font       = \slidetitlefont
  
\setarea{date}
  hshift*    = 5.5cm
  vshift     = 9cm
  height     = 22pt
  topskip    = 10pt
  background = white
  foreground = black
  top        = 5pt
  frame      = "width=.5em, corner=bevel, color=lightgrey"
  hpos       = rr

%
% Default step.
%
\setparameter step:
  everyvstep = "\quitvmode\llap{\itemsymbol\kern.5em}"
  vskip = 18pt

\def\i{\vskip14pt\quitvmode\llap{\itemsymbol\kern.5em}}

%
% Item symbols.
%
\newsymbol\itemsymbol[.6em,padding=0pt]
  {color darkred, + 1 0, + 0 1, + -1 0, fill,}
\newsymbol\mathsym[.5em,padding=0pt]
  {color darkred, 
  + 1 .5, + -1 .5, fill,}
\newsymbol\backsym[.5em,padding=0pt]
  {color darkred, move 1 0, + -1 .5, + 1 .5, fill,}
%
% Navigation to the appendices.
%
\def\Back#1{%
  \presentationonly{\usecolor{darkred}{\gotoB{#1}{\subtitlefont\backsym\kern.3em Back}}}%
  }
\def\To#1{%
  \presentationorhandout{\usecolor{darkred}{\gotoA{#1}{\subtitlefont\mathsym\kern.3em See why}}}
    {{\usecolor{darkred}{\subtitlefont\mathsym\kern.3em(See Appendices)}}}%
  }
%
% And some structure.
%
\def\section#1{%
  \def\sectiontitle{#1}%
  \createbookmark[nosubmenutext,open]{.5}{#1}%
  }


% \endinput
%%%%%%%%%%%%%%%%%%%%%%%%%%%%%%%%%%%%%%%%%%%%%%%%%%%%%%%%%%%%%%%
%                                                             %
% UNCOMMENT THE PREVIOUS LINE TO USE THIS FILE AS A TEMPLATE, %
% OR REMOVE EVERYTHING BELOW.                                 %
%                                                             %
%%%%%%%%%%%%%%%%%%%%%%%%%%%%%%%%%%%%%%%%%%%%%%%%%%%%%%%%%%%%%%%



\titleslide

\position{title}\Title
\position{author}\Author
\position{date}\Date

\endtitleslide





\section{Demonstration}

\slide[A simple assumption]


\i Some say the square root of 2 isn't rational.
\i Suppose it were.

\i[A] Then we could write it as this, where $a$ and $b$ are
integers without a common factor.

\i[B] But then we can also write this.
\i[C] And this.
\i[D] And finally this.
\i Which means that $a^2$ is even.

\endslide



\slide[Its consequences]


\i[visible="true"] So what?
\i So $a$ is even. Because only even numbers produce even squares.

\i[A] Being even means being expressible in the form $2k$, where $k$ is
any integer.
\i[B] And $(2k)^2$ square gives $4k^2$.

\i[C] Let's simplify.
\i Thus $b^2$ is even.
\i And $b$ is too.

\endslide



\slide[The problem]


\i[visible=true] And but so we said $a$ and $b$ have no common factor.

\i If both are even they do have a common factor: 2.
\i Which is absurd.

\i Thus, our basic assumption is false.
\i[A] There are no such $a$ and $b$.

\i The square root of 2 is irrational.
\i Too bad.

\endslide



\bye



